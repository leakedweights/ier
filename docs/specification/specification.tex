\documentclass[11pt]{article}
\usepackage[a4paper, margin=1in]{geometry}
\usepackage{enumitem}
\usepackage{titlesec}
\usepackage{authblk}
\usepackage{ragged2e}

\renewcommand{\Authfont}{\RaggedRight\normalsize}
\renewcommand{\Affilfont}{\RaggedRight\normalsize}
\titleformat*{\section}{\large\bfseries\RaggedRight}

\title{\RaggedRight\huge \textbf{Előzetes specifikáció}
 \newline \vspace*{6pt}\Large Okos mezőgazdasági rendszer}
\author{
Amer Jusuf (D0V869)
\newline
Friskó Dominika (ZPDBP7)
\newline
Sasvári Szabolcs Attila (TWOZG6)
\newline
\textbf{Váradi Kristóf (BP17IB)}}
\date{}

\begin{document}

\maketitle

\section{Áttekintés}

Az Okos Mezőgazdasági Rendszer egy olyan ágensekből álló környezet,
amely cselekvőképes robotokkal javítja a növénytermesztés hatékonyságát.
A rendszer három fő komponensből áll: egy terepanalízis drónból, amely a megművelt területek állapotát monitorozza,
egy ültető és betakarító ágensből, amely az optimális időpontban betakarítja a terményeket, illetve egy öntözési és tápanyagellátó robotból,
amely célzottan látja el a növényeket vízzel és tápanyagokkal.
Ezek az ágensek együttműködve, emberi beavatkozás nélkül
ellenőrzik és javítják a termények minőségét és mérséklik a mezőgazdasági tevékenységek környezeti hatását.

\subsection{Célkitűzések}
Mezőgazdasági adatok monitorozása,
termények automatizált ültetése, kezelése és betakarítása
a hozam növelése és az erőforrásigény csökkentése érdekében.
Kollaboráció a rendszer különböző szereplői között.
Öntözés, növényvédelmi kezelések célzott alkalmazása.
A fenntarthatóságának javítása a mezőgazdaságban,
az ökológiai lábnyom csökkentése.

\section{Ágensek}

Az alábbi ágensek egymással kommunikálva, emberi beavatkozás nélkül
végzik el a hatáskörükbe tartozó feladatokat, így az erőforrások (magok, t
panyagok, vegyszerek, stb.) biztosításán kívül a rendszer szereplői autonóm módon működnek.

\subsection{Terepanalízis Drón}

A megfigyelt mezőgazdasági területek monitorozása,
talaj és a növények állapotának elemzése,
a megszerzett adatok közlése a rendszer más szereplőivel.
Szárazság, tápanyaghiány vagy kártevőjelenlét detektálása.

\subsection{Automatizált Betakarító}

Az optimális ültetési és betakarítási időpont meghatározása, az ütemezett feladatok végrehajtása.

\subsection{Öntözési és Tápanyagellátó Robot}

Specifikus területek célzott öntözése és tápanyagellátása, kártevőirtás.
A víz- és tápanyaghasználat optimalizálása.

\end{document}
